\documentclass[a4paper,ngerman,12pt,bibtotoc]{scrartcl}

\usepackage[utf8]{inputenc}

\usepackage[ngerman]{babel}

\usepackage{amsmath, amsthm, amssymb, stmaryrd, color, graphicx, mathtools, mathrsfs}
\usepackage{setspace}
\usepackage{bussproofs}
\usepackage{array}
\usepackage{booktabs}
\usepackage{comment}
\usepackage{textcomp}
\usepackage{stmaryrd}

\usepackage[protrusion=true,expansion=true]{microtype}

\usepackage{lmodern}
\usepackage{tabto}

\usepackage[backend=bibtex,style=alphabetic]{biblatex}
\usepackage[babel]{csquotes}
\bibliography{literatur}

\usepackage{titling}

\usepackage[all]{xy}

\usepackage[colorlinks=true, linkcolor=blue, urlcolor=blue, citecolor=blue]{hyperref}
\usepackage{cleveref}			%Referenzen mit Name


\usepackage{algorithm}
\usepackage{algpseudocode}
\algrenewcommand{\algorithmiccomment}[1]{\hskip3em$\slash\slash$ #1}
\newcommand{\LineFor}[2]{\State\algorithmicfor\ {#1}\ \algorithmicdo\ {#2} \algorithmicend\ \algorithmicfor}


\setlength\parskip{\medskipamount}
\setlength\parindent{0pt}

\theoremstyle{definition}
\newtheorem{defn}{Definition}[section]
\newtheorem{axiom}[defn]{Axiom}
\newtheorem{bsp}[defn]{Beispiel}

\theoremstyle{plain}

\newtheorem{prop}[defn]{Proposition}
\newtheorem{motto}[defn]{Motto}
\newtheorem{ueberlegung}[defn]{Überlegung}
\newtheorem{lemma}[defn]{Lemma}
\newtheorem{kor}[defn]{Korollar}
\newtheorem{hilfsaussage}[defn]{Hilfsaussage}
\newtheorem{satz}[defn]{Satz}

\theoremstyle{remark}
\newtheorem{erin}[defn]{Erinnerung}
\newtheorem{bem}[defn]{Bemerkung}
\newtheorem{beob}[defn]{Beobachtung}
\newtheorem{aufg}[defn]{Aufgabe}

\clubpenalty=10000
\widowpenalty=10000
\displaywidowpenalty=10000

\newcommand{\IZ}{\mathbb{Z}}
\newcommand{\IQ}{\mathbb{Q}}
\newcommand{\IR}{\mathbb{R}}
\newcommand{\IC}{\mathbb{C}}
\newcommand{\IN}{\mathbb{N}}
\newcommand{\Ic}{\mathcal{I}}
\newcommand{\Jc}{\mathcal{J}}
\newcommand{\Hc}{\mathcal{H}}
\newcommand{\Tc}{\mathcal{T}}
\newcommand{\Sc}{\mathcal{S}}
\newcommand{\Oc}{\mathcal{O}}

% Nur für dieses Dokument %%%%%%%%%%%%%%%%%%%%%%%%%%%%%%%%%%%%%%

\newcommand{\ClientSet}{\mathscr{C}}
\newcommand{\FacilitySet}{\mathscr{F}}


\renewcommand*\theenumi{\alph{enumi}}
\renewcommand{\labelenumi}{(\theenumi)}

\setcounter{tocdepth}{2}


\usepackage{todonotes}


% DOCUMENT %%%%%%%%%%%%%%%%%%%%%%%%%%%%%%%%%%%%%%%%%%%%%%%%%%%%%

\begin{document}
	\author{Lukas Graf}
	\date{Letzte Aktualisierung: \today}
	
	\selectlanguage{ngerman}
	\thispagestyle{empty}
	
	
	\begin{titlepage}\center
	\textsc{\LARGE Universität Augsburg}\\[1.5cm]
	
	\textsc{\Large Institut für Mathematik}\\[2.5cm]
	
	% Title
	{\Large Ausarbeitung \\[1cm]}
	zum Programmierprojekt\\[1.5cm]
	{\huge ...}
		
	
	\vfill
	
	% Author and supervisor
	\begin{minipage}{0.4\textwidth}
		\begin{flushleft} \large
			\emph{von:}\\
			Lukas \textsc{Graf}
		\end{flushleft}
	\end{minipage}
	\begin{minipage}{0.4\textwidth}
		\begin{flushright} \large
			\emph{Betreut von:} \\
			Prof. Dr. Tobias \textsc{Harks}
		\end{flushright}
	\end{minipage}
	
	\end{titlepage}

% CONTENT %%%%%%%%%%%%%%%%%%%%%%%%%%%%%%%%%%%%%%%%%%%

	

	\section{Problemdefinitionen}
	
	\subsection{Capacitated Location Routing Problem}
	
	Eine Instanz des \textbf{Capacitated Location Routing Problems (CLR)} ist gegeben durch:
	\begin{itemize}
		\item einen ungerichteten, zusammenhängenden Graphen $G =(V,E)$,
		\item einer Partition der Knoten in Klienten $\ClientSet$ und Depots $\FacilitySet$,
		\item einer metrischen Kostenfunktion auf den Kanten $c: E \to \IR_{geq 0}$,
		\item Eröffnungskosten für die Fabriken $\phi: \FacilitySet \to \IR_{\geq 0}$,
		\item Bedarfen der Klienten $d: \ClientSet \to \IR_{\geq 0}$
		\item und einer einheitlichen Kapazität $u > 0$ für die Fahrzeuge.		
	\end{itemize}
	Zulässige Lösungen bestehen aus
	\begin{itemize}
		\item einer Teilmenge $F \subseteq \FacilitySet$ von eröffneten Fabriken
		\item und einer Menge von Touren $\Tc = \{T_1, \dots, T_k\}$,
	\end{itemize}
	sodass gilt:
	\begin{itemize}
		\item Zu jeder Tour gibt es ein eröffnetes Fabriken $f \in F$, an dem diese startet und endet.
		\item Alle Touren zusammen erfüllen alle Bedarfe der Klienten.
		\item Keine der Touren übersteigt die Kapazität $u$.
	\end{itemize}
	Das Optimierungsziel ist es die Gesamtkosten für das Eröffnen der Fabriken und die gefahrenen Touren zu minimieren, also die Minimierung der Kostenfunktion
		\[\sum_{T\in\Tc} c(T) + \sum_{f\in F}\phi(f)\footnote{Überladung der Funktion $c$} \]

	\subsection{Capacitated Location Routing with Hard Facility Capacities}
	
	Eine Instanz von \textbf{Capacitated Location Routing with Hard Facility Capacities (CLRHFC)} ist gegeben durch:
	\begin{itemize}
		\item einen ungerichteten, zusammenhängenden Graphen $G =(V,E)$,
		\item einer Partition der Knoten in Klienten $\ClientSet$ und Depots $\FacilitySet$,
		\item einer metrischen Kostenfunktion auf den Kanten $c: E \to \IR_{geq 0}$,
		\item Eröffnungskosten für die Fabriken $\phi: \FacilitySet \to \IR_{\geq 0}$,
		\item Bedarfen der Klienten $d: \ClientSet \to \IR_{\geq 0}$,
		\item Kapazitäten der Fabriken $l: \FacilitySet \to \IR_{\geq 0}$
		\item und einer einheitlichen Kapazität $u > 0$ für die Fahrzeuge.		
	\end{itemize}
	Zulässige Lösungen bestehen aus
	\begin{itemize}
		\item einer Teilmenge $F \subseteq \FacilitySet$ von eröffneten Fabriken
		\item und einer Menge von Touren $\Tc = \{T_1, \dots, T_k\}$,
	\end{itemize}
	sodass gilt:
	\begin{itemize}
		\item Zu jeder Tour gibt es ein eröffnetes Fabriken $f \in F$, an dem diese startet und endet.
		\item Alle Touren zusammen erfüllen alle Bedarfe der Klienten.
		\item Keine der Touren übersteigt die Kapazität $u$.
		\item Die Kapazitäten der Fabriken werden eingehalten.
	\end{itemize}
	Das Optimierungsziel ist es die Gesamtkosten für das Eröffnen der Fabriken und die gefahrenen Touren zu minimieren, also die Minimierung der Kostenfunktion
	\[\sum_{T\in\Tc} c(T) + \sum_{f\in F}\phi(f) \]

	
	\section{Visualisierung}
	
	\subsection{Der Algorithmus}
	
	\missingfigure{Schematische Darstellung des Algorithmus mit farblicher Kennzeichnung der Ausgabestellen}
	
	\subsection{title}
	
	\todo[inline]{Beschreibung der Visualisierungs-Klasse}
		
	
	\section{Anpassungen}
	
	\todo[inline]{Ideen und Probleme für Anpassungen}
	
	\todo[inline]{Beschreibung des angepassten Algorithmus}
	
	\todo[inline]{Untere Schranken}
	
	\todo[inline]{Heuristische Beurteilung}
	
	
	\listoftodos
	
	\newpage
	\nocite{*}
	\printbibliography		
			
\end{document}