\documentclass[a4paper,ngerman,12pt,bibtotoc]{scrartcl}

\usepackage[utf8]{inputenc}

\usepackage[ngerman]{babel}

\usepackage{amsmath, amsthm, amssymb, stmaryrd, color, graphicx, mathtools, mathrsfs}
\usepackage{setspace}
\usepackage{bussproofs}
\usepackage{array}
\usepackage{booktabs}
\usepackage{comment}
\usepackage{textcomp}
\usepackage{stmaryrd}

\usepackage{tikz}
\usetikzlibrary{shapes,arrows}

\usepackage[protrusion=true,expansion=true]{microtype}

\usepackage{lmodern}
\usepackage{tabto}

\usepackage[backend=bibtex,style=alphabetic]{biblatex}
\usepackage[babel]{csquotes}
\bibliography{literatur}

\usepackage{titling}

\usepackage[all]{xy}

\usepackage[colorlinks=true, linkcolor=blue, urlcolor=blue, citecolor=blue]{hyperref}
\usepackage{cleveref}			%Referenzen mit Name


\usepackage{algorithm}
\usepackage{algpseudocode}
\algrenewcommand{\algorithmiccomment}[1]{\hskip3em$\slash\slash$ #1}
\newcommand{\LineFor}[2]{\State\algorithmicfor\ {#1}\ \algorithmicdo\ {#2} \algorithmicend\ \algorithmicfor}


\setlength\parskip{\medskipamount}
\setlength\parindent{0pt}

\theoremstyle{definition}
\newtheorem{defn}{Definition}[section]
\newtheorem{axiom}[defn]{Axiom}
\newtheorem{bsp}[defn]{Beispiel}

\theoremstyle{plain}

\newtheorem{prop}[defn]{Proposition}
\newtheorem{motto}[defn]{Motto}
\newtheorem{ueberlegung}[defn]{Überlegung}
\newtheorem{lemma}[defn]{Lemma}
\newtheorem{kor}[defn]{Korollar}
\newtheorem{hilfsaussage}[defn]{Hilfsaussage}
\newtheorem{satz}[defn]{Satz}

\theoremstyle{remark}
\newtheorem{erin}[defn]{Erinnerung}
\newtheorem{bem}[defn]{Bemerkung}
\newtheorem{beob}[defn]{Beobachtung}
\newtheorem{aufg}[defn]{Aufgabe}

\clubpenalty=10000
\widowpenalty=10000
\displaywidowpenalty=10000

\newcommand{\IZ}{\mathbb{Z}}
\newcommand{\IQ}{\mathbb{Q}}
\newcommand{\IR}{\mathbb{R}}
\newcommand{\IC}{\mathbb{C}}
\newcommand{\IN}{\mathbb{N}}
\newcommand{\Ic}{\mathcal{I}}
\newcommand{\Jc}{\mathcal{J}}
\newcommand{\Hc}{\mathcal{H}}
\newcommand{\Tc}{\mathcal{T}}
\newcommand{\Sc}{\mathcal{S}}
\newcommand{\Oc}{\mathcal{O}}

% Nur für dieses Dokument %%%%%%%%%%%%%%%%%%%%%%%%%%%%%%%%%%%%%%

\newcommand{\ClientSet}{\mathscr{C}}
\newcommand{\FacilitySet}{\mathscr{F}}

\newcommand{\OPT}{\mathrm{OPT}}
\newcommand{\CLR}{\mathrm{CLR}}
\newcommand{\MST}{\mathrm{MST}}
\newcommand{\ULF}{\mathrm{ULF}}

\renewcommand*\theenumi{\alph{enumi}}
\renewcommand{\labelenumi}{(\theenumi)}

\setcounter{tocdepth}{2}


\usepackage{todonotes}


% DOCUMENT %%%%%%%%%%%%%%%%%%%%%%%%%%%%%%%%%%%%%%%%%%%%%%%%%%%%%

\begin{document}
\author{Lukas Graf}
\date{Letzte Aktualisierung: \today}

\selectlanguage{ngerman}
\thispagestyle{empty}


\begin{titlepage}\center
	\textsc{\LARGE Universität Augsburg}\\[1.5cm]
	
	\textsc{\Large Institut für Mathematik}\\[2.5cm]
	
	% Title
	{\Large Ausarbeitung \\[1cm]}
	zum Programmierprojekt\\[1.5cm]
	{\huge ...}
		
	
	\vfill
	
	% Author and supervisor
	\begin{minipage}{0.4\textwidth}
		\begin{flushleft} \large
			\emph{von:}\\
			Lukas \textsc{Graf}
		\end{flushleft}
	\end{minipage}
	\begin{minipage}{0.4\textwidth}
		\begin{flushright} \large
			\emph{Betreut von:} \\
			Prof. Dr. Tobias \textsc{Harks}
		\end{flushright}
	\end{minipage}
	
\end{titlepage}

% CONTENT %%%%%%%%%%%%%%%%%%%%%%%%%%%%%%%%%%%%%%%%%%%

	
\section*{\glqq Abstract\grqq}

\todo[inline]{Zusammenfassung/Überblick der Arbeit}


\section{Das Capacitated Location Routing Problem (CLR)}

	\subsection{Problemdefinition}

Eine Instanz des \textbf{Capacitated Location Routing Problems (CLR)} ist gegeben durch:
\begin{itemize}
	\item einen ungerichteten, zusammenhängenden Graphen $G =(V,E)$,
	\item einer Partition der Knoten in Klienten $\ClientSet$ und Depots $\FacilitySet$,
	\item einer metrischen Kostenfunktion auf den Kanten $c: E \to \IR_{geq 0}$,
	\item Eröffnungskosten für die Fabriken $\phi: \FacilitySet \to \IR_{\geq 0}$,
	\item Bedarfen der Klienten $d: \ClientSet \to \IR_{\geq 0}$
	\item und einer einheitlichen Kapazität $u > 0$ für die Fahrzeuge.		
\end{itemize}
Zulässige Lösungen bestehen aus
\begin{itemize}
	\item einer Teilmenge $F \subseteq \FacilitySet$ von eröffneten Fabriken
	\item und einer Menge von Touren $\Tc = \{T_1, \dots, T_k\}$,
\end{itemize}
sodass gilt:
\begin{itemize}
	\item Zu jeder Tour gibt es ein eröffnetes Fabriken $f \in F$, an dem diese startet und endet.
	\item Alle Touren zusammen erfüllen alle Bedarfe der Klienten.
	\item Keine der Touren übersteigt die Kapazität $u$.
\end{itemize}
Das Optimierungsziel ist es die Gesamtkosten für das Eröffnen der Fabriken und die gefahrenen Touren zu minimieren, also die Minimierung der Kostenfunktion
	\[\sum_{T\in\Tc} c(T) + \sum_{f\in F}\phi(f)\footnote{Überladung der Funktion $c$} \]


	\subsection{Der Algorithmus}

\begin{figure}[h]
	\begin{tiny}
		
\tikzstyle{Absch} = [rectangle, draw, 
    text centered]
\tikzstyle{keinBeweis} = [rectangle, fill=gray!35, 
    text centered]    

\tikzstyle{BewTeil} = []


\tikzstyle{Box} = [rectangle, draw, 
text centered, rounded corners]
\tikzstyle{Alg} = [rectangle, draw, fill=gray!50, text centered]
\tikzstyle{Text} = [ 
text centered]

\tikzstyle{line} = [draw, -latex']
\tikzstyle{line2} = [draw]


\begin{tikzpicture}[node distance = 5em, auto]

% EINGABE:
\node [Box, text width=20em] (input) {\textbf{Input:} \\ CLR-Instanz $((\ClientSet\cup\FacilitySet,E),c,\phi,d,u)$};
\node [below of=input] (under-input) {};

% ULF-Instanz:
\node [Box, left of=under-input, text width=12em, node distance=7em] (ULF) {\textbf{ULF-Instanz:} \\ $((\ClientSet\cup\FacilitySet,E),c,\phi,d,u)$};
\node [Text, left of=ULF, text width=12em, node distance=15em] (ULF-lower-bound) {$\leadsto$ Untere Schranke: $\OPT(\CLR) \geq \OPT(\ULF)$};
\node [Alg, below of=ULF, text width=12em] (ULF-solving) {löse approximativ (mit Greedy)};

% MST-Instanz:
\node [Box, right of=under-input, text width=12em, node distance=7em] (MST) {\textbf{MST-Instanz:} \\ $((\ClientSet\cup\FacilitySet,E),c,\phi,d,u)$};
\node [Text, right of=MST, text width=12em, node distance=15em] (MST-lower-bound) {$\leadsto$ Untere Schranke: $\OPT(\CLR) \geq \OPT(\MST)$};
\node [Alg, below of=MST, text width=12em] (MST-solving) {löse exakt (mit ???)};

% MERGE-Phase:
\node [Alg, below of=under-input, node distance=10em] (merge) {Merge-Phase:};

\node [Text, below of=merge, node distance=2em] (m-facilities) {Eröffne in $\ULF$ oder $\MST$ verwendete Facilities};

\node [Text, below of=m-facilities, node distance=2em] (m-big-clients) {Verbinde Klienten $v$ mit $d(v)\geq u$ mit nächstliegender offener Fabrik};

\node [Text, below of=m-big-clients, node distance=2em] (m-small-clients) {Für jede durch die $\MST$-Lsg eröffnete Farbik $f$ tue...};

\node [Box, below of=m-small-clients, node distance=10em] (m-relieve) {
	\begin{minipage}{20em}
		\begin{algorithmic}
		\While{$D_f > 0$}
		\State Finde $v \in S_f$ mit $D_v > u$ und f.a. Kinder $w$ von $v$: $D_w \leq u$
		\State Zerlege $S_v$ in Teilbäume ...
		\EndWhile
		\end{algorithmic}
	\end{minipage}
};

\node [Text, below of=m-relieve, node distance=10em] (m-remaining) {Mache rest...};


% AUSGABE:
\node [Box, below of=m-remaining, node distance=4em] (output) {Lösung};

\path [line] (input.south) -- (ULF.north);    
\path [line] (input.south) -- (MST.north); 

\path [line] (ULF-solving.south) -- (merge.north);
\path [line] (MST-solving.south) -- (merge.north);

\path [line] (m-remaining.south) -- (output.north);

\end{tikzpicture}

	\end{tiny}
	\caption{Schematische Darstellung des Algorithmus für CLR}
\end{figure}

\subsection{Visualisierung}

\todo[inline]{Beschreibung der Klasse zur Visualisierung}

	

\section{Capacitated Location Routing with Hard Facility Capacities}

	\subsection{Problemdefinition}

Eine Instanz von \textbf{Capacitated Location Routing with Hard Facility Capacities (CLRHFC)} ist gegeben durch:
\begin{itemize}
	\item eine Instanz $(G=(\ClientSet\cup\FacilitySet,E), c,\phi,d,u)$ von CLR
	\item und zusätzlich Kapazitäten der Fabriken $l: \FacilitySet \to \IR_{\geq 0}$.
\end{itemize}
Zulässige Lösungen sind Lösungen der zugrunde liegenden CLR-Instanz, die zudem die Kapazitätsschranken der Fabriken einhalten.

Das Optimierungsziel weiterhin die Minimierung der Kostenfunktion der CLR-Instanz.


	\subsection{Lösungsansätze}

\todo[inline]{Ideen und Probleme für Anpassungen}

	\subsection{Algorithmus}
	
\todo[inline]{Beschreibung des angepassten Algorithmus}

	\subsection{Analyse des Algorithmus}

\todo[inline]{Untere Schranken}

\todo[inline]{Heuristische Beurteilung}


\newpage	
\listoftodos

\newpage
\nocite{*}
\printbibliography		
			
\end{document}